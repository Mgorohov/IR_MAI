\documentclass[12pt,a4paper]{article}

\usepackage[a4paper,margin=2cm]{geometry}
\usepackage[T2A]{fontenc}
\usepackage[utf8]{inputenc}
\usepackage[russian]{babel}
\usepackage{lmodern}
\usepackage{graphicx}          % Картинки
\usepackage{float}

\usepackage{microtype}
\usepackage{setspace}
\setstretch{1.3}

\usepackage{hyperref}
\hypersetup{
  colorlinks=false,
  linkcolor=black,
  urlcolor=black,
  citecolor=black
}

\usepackage{booktabs}
\usepackage{tabularx}

\usepackage{enumitem}
\setlist[itemize]{leftmargin=1.5em, itemsep=0.2em, topsep=0.3em}
\setlist[enumerate]{leftmargin=1.5em, itemsep=0.2em, topsep=0.3em}

\usepackage{listings}
\usepackage{xcolor}

\definecolor{codebackground}{rgb}{0.97,0.97,0.97}

\lstdefinestyle{bash}{
  language=bash,
  basicstyle=\ttfamily\small,
  breaklines=true,
  columns=fullflexible,
  frame=lines,
  framerule=0.5pt,
  backgroundcolor=\color{codebackground},
  xleftmargin=0.5em,
  xrightmargin=0.5em,
  showstringspaces=false
}

\lstdefinestyle{cpp}{
  language=C++,
  basicstyle=\ttfamily\small,
  breaklines=true,
  frame=lines,
  backgroundcolor=\color{codebackground},
  showstringspaces=false,
  xleftmargin=0.5em,
  xrightmargin=0.5em
}

\lstset{style=bash}

\setlength{\parindent}{1.25cm}
\setlength{\parskip}{0.2em}

\newcommand{\file}[1]{\texttt{#1}}
\newcommand{\code}[1]{\texttt{#1}}

\begin{document}

\begin{titlepage}
\centering
{\large МОСКОВСКИЙ АВИАЦИОННЫЙ ИНСТИТУТ \par}
{\large (НАЦИОНАЛЬНЫЙ ИССЛЕДОВАТЕЛЬСКИЙ УНИВЕРСИТЕТ) \par}
\vspace{1.6cm}
{\large Институт №8 «Компьютерные науки и прикладная математика» \par}

\vfill

{\LARGE \textbf{Отчет по лабораторным работам} \par}
\vspace{0.4cm}
{\Large по курсу «Информационный поиск» \par}

\vfill

\begin{flushright}
Выполнил: Горохов Михаил Сергеевич\\
Группа: М8О-409Б-22\\
Преподаватель: Кухтичев Антон Алексеевич
\end{flushright}

\vfill

{Москва, 2025}
\end{titlepage}

\tableofcontents

\newpage

\section{Введение}

\subsection{Цель работы}

Целью данной работы является разработка полнофункциональной поисковой системы по собственному корпусу текстовых документов, включающей все основные этапы обработки информации: сбор данных, хранение, индексацию, нормализацию текста и реализацию методов поиска.

\subsection{Задачи}

В ходе выполнения работы решены следующие задачи:

\begin{enumerate}
  \item Разработка автоматизированного веб-краулера для сбора текстового корпуса из открытых источников (Project Gutenberg).
  \item Реализация хранилища документов на базе MongoDB с поддержкой управления большими объемами данных.
  \item Разработка модуля токенизации и нормализации текста на основе алгоритма Портера.
  \item Создание индексной структуры (инвертированный индекс) на собственных структурах данных без использования STL.
  \item Реализация булева поиска с поддержкой операций конъюнкции (AND) и отрицания (NOT).
  \item Проведение статистического анализа корпуса на соответствие закону Ципфа.
  \item Разработка интерфейсов взаимодействия: командной строки (CLI) и веб-приложения на Flask.
\end{enumerate}

\subsection{Структура проекта}

Система состоит из следующих компонентов:

\begin{itemize}
  \item Python-скрипты для сбора документов, загрузки в базу данных и реализации пользовательских интерфейсов
  \item Библиотека C++ с реализацией основных алгоритмов обработки текста
  \item База данных MongoDB для хранения полного текста документов
  \item Веб-интерфейс на HTML/Python для взаимодействия с системой
\end{itemize}

\section{Подготовка корпуса документов}

\subsection{Источник данных}

В качестве источника текстовых документов использован проект Project Gutenberg, предоставляющий более 70 тысяч свободно распространяемых электронных книг в различных форматах.

\subsubsection{Характеристики источника}

\begin{table}[h]
\centering
\small
\begin{tabularx}{\textwidth}{@{}lX@{}}
\toprule
Характеристика & Описание \\
\midrule
Тип документов & Электронные книги и литературные произведения \\
Язык & Английский \\
Формат & Простой текст (Plain text), кодировка UTF-8 \\
Лицензирование & Общественное достояние (Public Domain) \\
Доступность & Свободный доступ через веб-интерфейс \\
Объем одного документа & 10 тысяч до 500 тысяч слов \\
\bottomrule
\end{tabularx}
\caption{Характеристики источника данных}
\label{tab:source}
\end{table}

\subsection{Структура сырого документа}

Типичный документ Project Gutenberg содержит следующие элементы:

\begin{itemize}
  \item Заголовок проекта Гутенберга с лицензионной информацией
  \item Метаинформация произведения (название, автор, дата создания)
  \item Основной текст произведения
  \item Сноски и комментарии издателя
  \item Оглавление или содержание
\end{itemize}

\subsection{Подготовка корпуса}

Процесс подготовки корпуса включает следующие этапы:

\begin{enumerate}
  \item Загрузка текстовых файлов из веб-источника
  \item Удаление служебных блоков Project Gutenberg
  \item Проверка кодировки и структуры файлов
  \item Сохранение документов в MongoDB с метаданными
\end{enumerate}

\subsection{Статистика корпуса}

\begin{table}[h]
\centering
\small
\begin{tabularx}{\textwidth}{@{}Xr@{}}
\toprule
Параметр & Значение \\
\midrule
Количество документов & 40 000 \\
Суммарный объем (raw) & примерно 17 GB \\
Выделенный текст & 25,12 миллиардов символов \\
Средний размер документа & 628 000 символов \\
Уникальные термины & 300 000 - 500 000 \\
Общее количество токенов & примерно 2 миллиарда \\
Средняя длина токена & 7 символов \\
\bottomrule
\end{tabularx}
\caption{Статистика корпуса из 40 000 документов}
\label{tab:corpus}
\end{table}

\section{Архитектура системы}

\subsection{Общее описание}

Система построена на модульной архитектуре с разделением функциональности между различными слоями. Основные компоненты взаимодействуют следующим образом:

\begin{table}[h]
\centering
\begin{tabularx}{\textwidth}{@{}lll@{}}
\toprule
Python Layer & Core Engine (C++) & Storage \\
\midrule
CLI Interface & Tokenization & MongoDB \\
Web Service & Stemming & \\
Crawler & Boolean Search & \\
\bottomrule
\end{tabularx}
\caption{Слои архитектуры системы}
\label{tab:arch}
\end{table}

\subsection{Компоненты}

\subsubsection{Компонент сбора данных}

Скрипт \file{download\_documents.py} реализует функции автоматизированного веб-краулера:

\begin{itemize}
  \item Подключение к веб-ресурсу Project Gutenberg
  \item Парсинг HTML-страниц книг
  \item Загрузка текстовых файлов
  \item Сохранение документов с метаданными
  \item Вежливый краулинг с задержками между запросами
\end{itemize}

\subsubsection{Хранилище документов}

MongoDB используется для хранения полных текстов документов с следующей структурой:

\begin{lstlisting}[style=bash]
db.documents.insertOne({
  _id: ObjectId(...),
  title: "Pride and Prejudice",
  author: "Jane Austen",
  url: "https://www.gutenberg.org/ebooks/1342",
  content: "It is a truth universally acknowledged...",
  download_date: ISODate("2025-01-15")
})
\end{lstlisting}

\subsubsection{Ядро обработки (C++)}

Реализует функции индексации и поиска с использованием собственных структур данных:

\begin{itemize}
  \item Собственные классы String, Vector и HashMap
  \item Модули токенизации, стемминга и индексации
  \item Компиляция в библиотеку \file{libir\_system.so}
  \item Интеграция с Python через ctypes
\end{itemize}

\section{Реализация поисковой системы}

\subsection{Этап первый: Токенизация}

\subsubsection{Описание алгоритма}

Токенизация преобразует входной текст в набор отдельных терминов для последующей обработки. Процесс включает следующие шаги:

\begin{enumerate}
  \item Преобразование текста в нижний регистр
  \item Выделение последовательностей буквенно-цифровых символов
  \item Использование знаков пунктуации и пробелов в качестве разделителей
  \item Формирование упорядоченного набора токенов
\end{enumerate}

\subsubsection{Пример токенизации}

\begin{lstlisting}[style=bash]
Входной текст:
"It's a beautiful day! The sun shines brightly."

Результат токенизации:
["it", "s", "a", "beautiful", "day", "the", 
 "sun", "shines", "brightly"]
\end{lstlisting}

\subsubsection{Производительность}

\begin{table}[h]
\centering
\small
\begin{tabularx}{\textwidth}{@{}Xr@{}}
\toprule
Метрика & Значение \\
\midrule
Документов обработано & 40 000 \\
Уникальные термины & 300 000 - 500 000 \\
Общее количество токенов & примерно 2 миллиарда \\
Средняя длина токена & 7 символов \\
Время обработки & 90 - 120 секунд \\
Скорость обработки & примерно 30 000 KB/s \\
\bottomrule
\end{tabularx}
\caption{Статистика этапа токенизации}
\label{tab:tokenization}
\end{table}

\subsection{Этап второй: Нормализация}

\subsubsection{Алгоритм стемминга}

Нормализация текста приводит различные словоформы к единому корню:

\begin{itemize}
  \item running к run
  \item jumped к jump
  \item beautifully к beauti
\end{itemize}

Используется упрощенный алгоритм Портера с удалением известных суффиксов.

\subsubsection{Преимущества стемминга}

\begin{enumerate}
  \item Повышение полноты поиска (recall): поиск по book находит books и booking
  \item Уменьшение размера словаря и индекса
  \item Улучшение качества поиска за счет нормализации словоформ
\end{enumerate}

\subsection{Этап третий: Построение индекса}

\subsubsection{Структура инвертированного индекса}

Инвертированный индекс представляет отображение терминов на список документов:

\begin{table}[h]
\centering
\small
\begin{tabularx}{\textwidth}{@{}lXr@{}}
\toprule
Термин & Документы & Частота \\
\midrule
book & Doc1, Doc3, Doc5, Doc12 & 4 \\
love & Doc2, Doc5, Doc8, Doc15, Doc20 & 5 \\
story & Doc1, Doc2, Doc3, Doc4, Doc5 & 5 \\
\bottomrule
\end{tabularx}
\caption{Пример структуры инвертированного индекса}
\label{tab:inverted}
\end{table}

\subsubsection{Реализация структур данных}

Индекс реализован на основе собственных структур данных C++:

\begin{itemize}
  \item HashMap для хранения соответствия терминов и списков документов
  \item PostingList для управления списком документов с частотой
  \item PostingEntry для хранения информации о документе
\end{itemize}

\subsubsection{Характеристики индекса}

\begin{table}[h]
\centering
\small
\begin{tabularx}{\textwidth}{@{}Xr@{}}
\toprule
Параметр & Значение \\
\midrule
Размер словаря & примерно 400 000 терминов \\
Средний размер PostingList & примерно 10 документов \\
Объем индекса в памяти & примерно 2 GB \\
Время построения индекса & примерно 2 минуты \\
\bottomrule
\end{tabularx}
\caption{Характеристики инвертированного индекса}
\label{tab:index_char}
\end{table}

\subsection{Этап четвертый: Анализ по закону Ципфа}

\subsubsection{Описание закона}

Закон Ципфа описывает распределение частот слов в естественном языке:

\[f(r) = \frac{C}{r^\alpha}\]

где r является рангом термина, f(r) частотой, C константой, и альфа показателем степени.

\subsubsection{Результаты анализа}

На основе построенного индекса вычисляются частоты терминов по рангам. Результаты сохраняются в файл \file{data/documents/zipf.csv}:

\begin{lstlisting}[style=bash]
rank,frequency,zipf_approximation
1,45230,45230.5
2,23450,22615.2
3,15120,15076.8
4,11280,11288.6
\end{lstlisting}

\begin{figure}[H]
  \centering
  \includegraphics[width=0.8\linewidth]{zipf.png} % Замените на путь к вашему графику
  \caption{График распределения частот терминов по рангам (Закон Ципфа)}
  \label{fig:WEB}
\end{figure}

Анализ показывает, что распределение частот в корпусе Project Gutenberg соответствует закону Ципфа, что подтверждает качество и естественность текстовых данных.

\subsection{Этап пятый: Булев поиск}

\subsubsection{Поддерживаемые операции}

Система реализует следующие операции булева поиска:

\begin{enumerate}
  \item Простой поиск по одному термину
  \begin{itemize}
    \item Возвращает все документы, содержащие данный термин
  \end{itemize}
  
  \item Конъюнкция (AND)
  \begin{itemize}
    \item Несколько терминов через пробел
    \item Возвращает документы, содержащие все термины
    \item Реализуется пересечением PostingList
  \end{itemize}
  
  \item Отрицание (NOT)
  \begin{itemize}
    \item Исключение терминов из результатов
    \item Поддержка синтаксиса NOT и дефиса
    \item Реализуется разностью множеств
  \end{itemize}
\end{enumerate}

\subsubsection{Примеры поисковых запросов}

\begin{table}[h]
\centering
\small
\begin{tabularx}{\textwidth}{@{}lX@{}}
\toprule
Запрос & Результат \\
\midrule
love & Документы со словом love \\
great book & Документы с обоими словами great и book \\
book NOT chapter & Документы со словом book без chapter \\
old man NOT young & Документы с old и man без young \\
\bottomrule
\end{tabularx}
\caption{Примеры поисковых запросов}
\label{tab:queries}
\end{table}

\section{Интерфейсы взаимодействия}

\subsection{Интерфейс командной строки}

\subsubsection{Описание}

Интерфейс командной строки предоставляет интерактивное взаимодействие с поисковой системой в режиме реального времени.

\subsubsection{Запуск}

\begin{lstlisting}[style=bash]
$ python3 scripts/cli_search.py
Loading index from MongoDB...
Index loaded: 40000 documents, 350000 terms
Enter search query (or 'quit' to exit):
\end{lstlisting}

\subsubsection{Примеры использования}

\begin{lstlisting}[style=bash]
Enter search query: book
Found 2350 documents
1. Pride and Prejudice
2. Jane Eyre
3. The Great Gatsby

Enter search query: love story
Found 145 documents
1. Romeo and Juliet
2. Jane Eyre

Enter search query: book NOT chapter
Found 789 documents
\end{lstlisting}

\subsection{Веб-интерфейс}

\subsubsection{Описание}

Веб-приложение на Flask предоставляет графический интерфейс с поддержкой различных браузеров.

\subsubsection{Запуск сервера}

\begin{lstlisting}[style=bash]
$ python3 scripts/web_service.py
Running on http://127.0.0.1:5000
\end{lstlisting}

\begin{figure}[H]
  \centering
  \includegraphics[width=0.8\linewidth]{WEB.png}
  \includegraphics[width=0.8\linewidth]{WEB_2.png}
  \caption{Web приложение}
  \label{fig:WEB}
\end{figure}

\subsubsection{Структура приложения}

Веб-приложение содержит следующие страницы:

\begin{itemize}
  \item Главная страница с полем ввода запроса
  \item Страница результатов с списком найденных документов
\end{itemize}

\subsubsection{Функциональность}

\begin{enumerate}
  \item Ввод поискового запроса в текстовое поле
  \item Отправка запроса на сервер
  \item Вывод результатов поиска
  \item Возможность просмотра полного текста документа
\end{enumerate}

\section{Выводы}

\subsection{Полученные результаты}

В ходе выполнения работы разработана функциональная поисковая система, включающая следующие компоненты:

\begin{enumerate}
  \item Автоматизированный краулер для сбора текстовых данных
  \item Хранилище на базе MongoDB для управления документами
  \item Ядро обработки текста на языке C++
  \item Инвертированный индекс с поддержкой быстрого поиска
  \item Булев поиск с операциями AND и NOT
  \item Два интерфейса для взаимодействия пользователя
\end{enumerate}

\subsection{Практические навыки}

Работа над проектом позволила получить опыт в следующих областях:

\begin{itemize}
  \item Обработка больших объемов текстовых данных
  \item Реализация классических алгоритмов информационного поиска
  \item Проектирование собственных структур данных
  \item Интеграция компонентов на различных языках программирования
  \item Статистический анализ текстовых данных
  \item Разработка веб-приложений
\end{itemize}

\subsection{Возможности развития}

Текущая реализация может быть расширена следующим функционалом:

\begin{itemize}
  \item Ранжирование результатов на основе TF-IDF и BM25
  \item Автодополнение при вводе запроса
  \item Кеширование результатов для оптимизации
  \item Параллельная обработка для масштабирования
\end{itemize}

\subsection{Заключение}

Разработанная система демонстрирует применение теоретических основ информационного поиска при построении практической системы на собственном текстовом корпусе. Проект успешно объединяет высокопроизводительное ядро на языке C++ с удобными интерфейсами на Python и веб-платформе, обеспечивая эффективный баланс между производительностью и удобством использования.

\end{document}